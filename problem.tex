\section{Problem}
\subsection{Turnierbewertung}
In der Kampfsportart Taekwondo gibt es mehrere Teilbereiche, einer der bekanntesten ist das Formenlaufen oder auch Poomsae genannt.
Ein Sportler präsentiert dabei eine vorgegebene Formenabfolge und mehrere Schiedsrichter bewerten diese nach
Präzision, Ausdruck und Präsentation.
Da es sich um Einzelpräsentationen handelt, ist der Zeitablauf in so einem Turnier minutiös geplant.
Darum ist es zwingend erforderlich die Teilnehmer innerhalb kürzester Zeit zu bewerten.
Für die Coaches ist es wichtig nach dem ersten Durchgang die einzelnen vergebenen Punkte zu sehen, um ihren Sportlern
    noch letzte Tipps für den zweiten Durchgang zu geben.
Zudem ist es wichtig, die Anonymität der Schiedsrichter in Bezug auf die Punkte zu gewährleisten.
Viele der Turnierhallen haben kein WLAN oder aufgrund ihrer Baustruktur keinen Internet-Empfang, dennoch ist ein System
das Verkabelung benötigt alleine aufgrund der Anzahl an Kabeln und der Größe der Kampfflächen nicht praktikabel.
Daher werden auch noch viele Turniere händisch bewertet und berechnet.
	
\subsection{Unzureichende Lösungen}
Es gibt offiziell keine Poomsae-Bewertungssysteme die von der \emph{World Taekwondo Federation} anerkannt und käuflich erwerbbar sind.
Im deutschsprachigen Raum gibt es zwei private Lösungen, diese können lediglich ausgeliehen werden.
Die preislich günstig Variante benötigt einiges an Hardware (Bildschirme, Tablets, Kabel, uvm.), die der jeweilige Verein selbst stellen muss.
Zudem ist es bei diesem System schon öfters zu Ausfällen gekommen, was wiederum sofort eine enorme zeitliche
Verzögerung des gesamten Turniers bedeutet.
\\
Die andere Variante ist wesentlich stabiler und dadurch preislich bis um das Zweifache teurer.
Allerdings ist hier eine spezielle Hardware erforderlich, diese kann gegen einen Aufpreis auch ausgeliehen werden.

\section{Lösung}
In der heutigen Welt ist es unerlässlich auch solche Prozesse zu digitalisieren und somit schnelle,
    zuverlässige und anonyme Lösungen zu schaffen.
\\\\
Es ist naheliegend, dass das ganze System "Jeomsu\" erstellt werden musste:
Die Schiedsrichter benötigen eine sichere, stabile App mit der sie in Echtzeit die Sportler bewerten können.
Der Administrator einer einzelnen Kampfflächen muss die aktuellen Sportler auswählen und zudem noch die Anzeigen für die Coaches steuern können.
Ebenso wird eine Software benötigt, die nach der Präsentation die Endpunkte berechnet und anzeigen kann. 
Die erstellte App erfüllt all die benötigten Funktionen und ist zudem plattformunabhängig, somit ist sie für alle Geräte,
    sei es Tablet oder Handy verschiedener Hersteller kompatibel.
So bleibt es unkompliziert und Vereine und Veranstalter können sich die Mietkosten für die Hardware sparen.
Die App kann, falls Internet vorhanden ist, online mit der ebenfalls erstellten Software zur Berechnung der Endpunkte verbunden werden.
Ohne Internetanschluss ist auch eine lokale Kommunikation möglich, der dazu benötigte Server kann über den
    Postweg verschickt werden und verwendet werden.

\subsection{Alleinstellung}
Wie sich im vergangenen Jahr gezeigt hat, können unvorhergesehene Ereignisse auftreten, welche Turnierveranstalter*innen
    zwingen, mit neuen Ideen aufzuwarten.
Mit Jeomsu ergibt sich die einzigartige Situation, dass das gesamte System online zur Verfügung steht.
Daher ist das bisher Undenkbare erstmals möglich: auch in der schwierigen Situation, die aufgrund der Pandemie gerade
herrscht, ist es möglich online Turniere auszutragen.
Dieses Feature ist ein entscheidender Vorteil gegenüber anderen Systemen.
\\
Des Weiteren wird das System mit fundiertem Wissen im technischen wie auch im sportlichen Bereich entwickelt.
Aufgrund dieser Funktionen ist Jeomsu sowohl für kleine, als auch grosse Turniere geeignet und einsetzbar.

\section{Vision}
Turnierbewertungssoftware wird nicht nur im Taekwondo, sondern auch in anderen Sportarten wie beispielsweise im Eiskunstlauf,
    bei Gymnastik oder Geräteturnen.
Eben diese Sportarten haben auch ein international fixiertes Regelwerk und die Bewertungsschritte sind ähnlich.
Somit kann mit verhältnismässig geringem Aufwand die Software entsprechend angepasst werden und es eröffnet sich ein
    weitaus größerer Markt, auf dem sich unser System etablieren kann.
