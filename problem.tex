\section{Problem}

\subsection{Turnierbewertung}
In der Kampfsportart Taekwondo gibt es mehrere Teilbereiche, einer der bekanntesten ist das Formenlaufen oder auch Poomsae genannt.
Ein Sportler präsentiert dabei eine vorgegebene Formenabfolge und mehrere Schiedsrichter bewerten diese nach
Präzision, Ausdruck und Präsentation.
Da es sich um Einzelpräsentationen handelt, ist der Zeitablauf in so einem Turnier minutiös geplant.
Es ist daher zwingend erforderlich, die Teilnehmer innerhalb kürzester Zeit zu bewerten.
Für die Coaches ist es wichtig nach dem ersten Durchgang die einzelnen vergebenen Punkte zu sehen, um ihren Sportlern
    noch letzte Tipps für den zweiten Durchgang zu geben.
Zudem ist es wichtig, die Anonymität der Schiedsrichter in Bezug auf die Punkte zu gewährleisten.
Viele der Turnierhallen haben kein WLAN oder aufgrund ihrer Baustruktur keinen Mobilfunk-Empfang, dennoch ist ein System
das Verkabelung benötigt aufgrund der Anzahl an Kabeln und der Grösse der Kampfflächen nicht praktikabel.
Daher werden viele Turniere händisch bewertet und die Resultate anschliessend mühselig in ein Tabellenprogramm
eingetragen.

\subsection{Unzureichende Lösungen}

Es gibt kein Poomsae-Bewertungssystem das von der \emph{World Taekwondo Federation} zertifiziert und käuflich erwerbbar ist.
Im deutschsprachigen Raum gibt es zwei private Lösungen, diese können jedoch nur ausgeliehen werden.
%todo, plattform Windows Hardware
Die preislich günstigere Variante ist es schon öfters zu Ausfällen gekommen, was wiederum eine enorme zeitliche
Verzögerung des gesamten Turniers bedeutet.
\\
Die andere Variante ist wesentlich stabiler aber doppelt so teuer.
Allerdings ist hier eine spezielle Hardware, sprich ein Tablet des Herstellers Apple pro Schiedsrichter, erforderlich.
Diese kann gegen einen Aufpreis auch ausgeliehen werden.

\section{Lösung}

%todo, anonyme Lösungen
In der heutigen Welt ist es unerlässlich Prozesse zu digitalisieren und somit schnelle,
    zuverlässige und für die Schiedsrichter anonyme Lösungen zu schaffen.
\\
Es ist naheliegend, dass das ganze System \textit{Jeomsu} erstellt werden musste:
Die Schiedsrichter benötigen eine sichere, stabile App mit der sie in Echtzeit die Sportler bewerten können.
Der Administrator einer einzelnen Kampffläche muss die aktuellen Sportler auswählen und zudem die Anzeigen für die
Coaches steuern können.
Ebenso wird eine Software benötigt, die nach der Präsentation die Endpunkte berechnet und anzeigen kann. 
Die erstellte App erfüllt die benötigten Funktionen und ist zudem plattformunabhängig, somit ist sie für alle Geräte,
    sei es Tablet oder Handy verschiedener Hersteller kompatibel.
Dadurch entsteht eine günstige Lösung, Vereine und Veranstalter können ihre eigenen Tablets/Computer verwenden.
Dies spart Kosten und reduziert den zeitlichen Aufwand, der beim Einrichten eines Turniers anfällt.
Die App kann, falls Internet vorhanden ist, online mit der ebenfalls erstellten Software zur Berechnung der Endpunkte verbunden werden.
Ohne Internetanschluss ist auch eine lokale Kommunikation möglich, der dazu benötigte Server kann über den
    Postweg verschickt werden und verwendet werden.

\subsection{Alleinstellung}

Wie sich im vergangenen Jahr gezeigt hat, können unvorhergesehene Ereignisse auftreten, welche Turnierveranstalter
    zwingen, innovativ zu sein.
Mit \textit{Jeomsu} ergibt sich die einzigartige Situation, dass das gesamte System auch online zur Verfügung steht.
Daher ist das bisher Undenkbare erstmals möglich: auch in der schwierigen Situation, die aufgrund der Pandemie gerade
herrscht, ist es möglich online Turniere auszutragen.
Dieses Feature ist ein entscheidender Vorteil gegenüber anderen Systemen.
\\
Des Weiteren wird das System mit fundiertem Wissen im technischen wie auch im sportlichen Bereich entwickelt.
Aufgrund dieser Funktionen ist \textit{Jeomsu} sowohl für kleine, als auch grosse Turniere geeignet und einsetzbar.

\section{Vision}

Turnierbewertungssoftware wird nicht nur im Taekwondo, sondern auch in anderen Sportarten wie beispielsweise im Eiskunstlauf,
    bei Gymnastik oder Geräteturnen benötigt.
Diese Sportarten haben ein international fixiertes Regelwerk und die Bewertungsschritte sind denen von Taekwondo ähnlich.
Somit kann mit geringem Aufwand die Software für diese weiteren Sportarten angepasst werden.
Dadurch kann \textit{Jeomsu} einen deutlich grösseren Markt als die Konkurrenz bedienen.