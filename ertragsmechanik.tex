\section{Ertragsmechanik}
Wir gingen davon aus, dass wir für die Entwicklung des Turnierbewertungssystems ein Jahr einberechnen müssen und über diesen Zeitraum
    einen Softwareentwickler zu einem Jahreslohn von 85000€ beschäftigen würden.
Dazu kommen noch 15000€ für Werbezwecke, wodurch sich das geschätzte Kapital von 100000€ ergibt.
\\\\
Als nächstes überlegten wir uns mehrere Varianten für mögliche Einnahmequellen, so z.B. ob wir unser System vermieten und eine kostenpflichtige Support Hotline einrichten oder ob wir für die Benutzung unseres Systems Abo-Gebühren verlangen sollen.
Unabhängig von der Variante wollten wir ausserdem eine mobile App und Schulungen zur Benutzung des Systems anbieten.
\\\\
Die Kostenstruktur der beiden oben erwähnten Varianten umfasst lediglich fixe Kosten, und zwar Kosten für den Betrieb eines Webservers,
    Personalkosten für Wartungs- und Entwicklungsarbeiten, Mietkosten und Kosten für die benötigte Hardware.
Die Beträge dieser Kosten wären für beide Varianten gleich mit Ausnahme der Personalkosten.
Diese wären für die erste Variante doppelt so hoch, da wir mit zwei Angestellten statt nur einem (zu je 40\%) rechnen müssten,
    um die Verfügbarkeit der Support Hotline zu garantieren.
\\\\
Für die Kostenrechnung nahmen wir einen kalkulatorischen Zinssatz von 6\% an und eine Nutzungsdauer von fünf Jahren,
    während der es möglich sein sollte eine nächste Version des Systems zu entwickeln.
\\\\
Mittels der Gewinnrechnung passten wir einerseits unser Pricing an und stellten andererseits unsere Annahmen auf die Probe,
    um herauszufinden, ob und wie sich mit unserer Idee Geld verdienen lässt.
Wir mussten allerdings wir bald einsehen, dass unsere ursprüngliche Geschäftsidee – ein Turnierbewertungssystem für
    Taekwondo – unter den Annahmen, die wir getroffen hatten, nicht gewinnbringend ist.
Deshalb entschieden wir uns, unsere Idee auszuweiten und stattdessen ein Turnierbewertungssystem für Sportarten mit einem
    Bewertungsverfahren ähnlich dem beim Taekwondo ins Auge zu fassen.
\\\\
Schlussendlich entschieden wir uns dafür, unser System zwar zu vermieten, aber vorerst nur einen Mitarbeiter anzustellen
    und auf eine kostenpflichtige Support Hotline zu verzichten.
Für Fragen stünde jedoch jederzeit der Unternehmensgründer bereit, der auch wenn nötig bei der Wartung und Entwicklung
    des Systems mit anpacken könnte.
Eine weitere Möglichkeit Geld zu sparen sahen wir im Bereich der Mietkosten.
Indem wir diese Kosten um die Hälfte des ursprünglich angenommenen Betrages reduzierten, gelangten wir trotz einer
geringeren Anzahl an Turnieren (ursprünglich rechneten wir mit 300, neu mit 240) immer noch zu einem positiven Ergebnis.
Aus unserer Gewinnrechnung folgt ausserdem, dass für einen möglichen Gewinn die Einnahmen aus den Verkäufen der App unabdingbar sind,
    selbst wenn dies nicht unsere primäre Einnahmequelle ist.