\section{Marktpotenzial}\label{marktpotenzial}

Nach einem halben Jahr Entwicklungszeit ist ein erster funktionierender Prototyp vorhanden, dieser befindet sich aktuell in
der internen Testphase.
Sobald das interne Testing abgeschlossen ist, wird der Prototyp allen Taekwondo Vereinen zum Testen zur Verfügung gestellt.
Einzige Bedingung ist, dass die Vereine ein Feedback Formular ausfüllen welches zum Verbessern des Produkts genutzt werden kann.
Diese Aktion dient gleichzeitig als Werbeaktion für unser Produkt.
\\\\
Da die Datenlage bezüglich Anzahl Turniere sehr dünn ist, haben wir die Anzahl Veranstaltungen je Sportart geschätzt.
Als Referenz diente uns eine Publikation des Bundesamts für Sport. \parencite[S.~9]{baspo}
\\\\
Während der öffentlichen Testphase beträgt die Marktkapazität alle Taekwondo-Turnieren in deutschsprachigen Ländern
	ca. 80 Turniere pro Jahr.
Da es sich nur um ein kleines Team handelt ist es mit nur einem Prototyp-System schwierig, mehrere Turniere
	gleichzeitig abzuhalten, daher beschränkt sich das Marktpotenzial auf 52 Turniere pro Jahr, denn die meisten
	Turniere finden an einem Wochenende statt.
Es wäre wünschenswert, wenn wir in dieser Zeit einen Marktanteil von 30\% erreichen könnten, das entspricht 24
	Turnieren pro Jahr.
\\\\
Mit der Vision das System auf mehrere Sportarten auszuweiten könnten dann im zweiten Jahr deutlich mehr Turniere bewertet werden.
Gemäss unseren Schätzungen gibt es in den geplanten Sportarten Geräteturnen, Eiskunstlauf und Gymnastik total 330 Turniere pro Jahr.
Die gesamte Marktkapazität über alle Sportarten erhöht sich somit auf das Vierfache.

%todo, Einsatzbereite Systeme beschreiben
Mit mehreren einsatzbereiten Systemen können an einem Wochenende mehrere Turniere betreut werden.
So würde sich das Markpotential mit fünf einsatzbereiten System auf 264 Turniere erhöhen.
Bei den zusätzlich eingeführten Sportarten im zweiten Geschäftsjahr rechnen wir zu Beginn ebenfalls mit einem Marktanteil von 30\%.

Gemäss unserer Schätzung wird der Marktanteil von \textit{Jeomsu} der Taekwondo-Veranstaltungen bei 80\% im zweiten Geschäftsjahr liegen.
Dieses Ziel wird für die kommenden fünf Jahre auch für die neu eingeführten Sportarten angestrebt, mit weiteren
	einsatzfähigen Systemen könnte im Laufe der nächsten fünf Jahre dann einen Absatz von 240 Turnieren jährlich
	erreicht werden.


\section{Kennzahlen}

Wie im Absatz~\ref{marktpotenzial} beschrieben lässt sich der Erfolg des Systems in Turnieren, die pro Jahr
durchgeführt werden, beschreiben.
Als Erfolgszahlen die wir und anfänglich erhoffen legen wir fest:
\begin{itemize}
	\item Mindestens 24 ausgerichteten Turnieren im ersten Jahr
	\item Mindestens einen jährlichen Zuwachs um 300\% in den folgenden zwei Jahren bezogen auf die Anzahl der
	Turniere im ersten Jahr
\end{itemize}

\section{Verkaufs- und Marketingkanäle}

Damit möglichst viele von unserem Produkt erfahren, wollen wir unsere Software direkt den Taekwondo-Vereinen und
den Veranstaltern von Taekwondo-Turnieren vorstellen.
Hierfür ist zusätzlich ein Flyer für potenzielle Veranstalter anzudenken.
Da es keine offiziellen Systeme und Informationsseiten in der Taekwondo-Community bezüglich solcher Bewertungssysteme
gibt, wird sehr viel durch Mundpropaganda bekannt gemacht.
So hat sich auch schon der Prototyp herumgesprochen und ist bereits für mehrere Online-Turniere gebucht worden.
\\\\
Bei neuen Sportarten würden wir ähnlich vorgehen und uns gezielt Ansprechpersonen suchen, welche schon eine Weile diese
Sportart ausüben, so kann die Bewertungsapp optimal auf die spezifische Sportart angepasst werden.
Durch die dann entwickelte erste Version können bei Tests erste Kunden in anderen Sportarten gewonnen werden.
Zusätzlich zu der Vorstellung bei Vereinen bietet sich in bekannten Sportarten Werbung in Fachzeitschriften an.

\section{Kundengruppen}

Die Kundengruppe, für die das System entwickelt wird, sind alle Vereine, Obleute oder private Veranstalter von
	Turnieren für die Sportarten Taekwondo, Eiskunstlauf, Geräteturnen und Gymnastik betreffen.
Sie sind dafür verantwortlich, dass das Datenmanagement der Sportler, die Bewertung und auch die schlussendliche
	Auswertung der Punkte stabil und fehlerfrei vonstattengeht.
Das System \textit{Jeomsu} kommt dabei in all diesen Bereichen zum Einsatz, sowohl von dem Turnierleiter als auch von
	den Schiedsrichtern.