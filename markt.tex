\section{Marktpotential}\label{marktpotential}

Nach einem halben Jahr Entwicklungszeit ist ein erster funktionierender Prototyp vorhanden, dieser befindet sich aktuell in
der internen Testphase.
Sobald das interne Testing abgeschlossen ist, wird der Prototyp einer kleinen Kundegruppe zum Testen zur Verfügung gestellt.
Einzige Bedingung ist, dass die Kunden ein Feedback Formular ausfüllen welches zum Verbessern des Produkts genutzt
werden kann.
\\\\

Innerhalb dieser Zeit beträgt die Marktkapazität daher alle Taekwondo-Turnieren in deutschsprachigen Ländern, das
sind ca. 80 Turniere pro Jahr.
Da es sich nur um ein kleines Team handelt ist es mit nur einem Prototyp-System schwierig, mehrere Turniere
	gleichzeitig abzuhalten, daher beschränkt sich das Marktpotential auf 52 Turniere pro Jahr, denn die meisten
	Turniere finden an einem Wochenende statt.
Es wäre wünschenswert, wenn wir in dieser Zeit einen Marktanteil von 30\% erreichen könnten, das entspricht 32
	Turnieren pro Jahr.
\\\\

Mit der Vision das System auf mehrere Sportarten auszuweiten könnten schon im zweiten Jahr mehrere verschiedene
	Sportarten bewertet werden.
Recherchen haben ergeben, dass in den geplanten Sportarten Geräteturnen, Eiskunstlauf und Gymnastik pro
	Sportart ca. 100 Turniere jährlich stattfinden.
Die Marktkapazität erhöht sich somit auf das dreifache.
Mit mehreren einsatzbereiten Systemen können an einem Wochenende mehrere Turniere betreut werden.
So würde sich das Markpotential mit fünf einsatzbereiten System auf 260 Turniere erhöhen.
Durch die Neueinführung der zusätzlichen Sportarten im zweiten Geschäftsjahr rechnen wir ebenfalls mit einem
	Marktanteil von 30\% .
Mit dem zu diesem Zeitpunkt in der Taekwondo-Community bereits bekannten System könnten wir dort schon das Ziel von
	80\% Marktanteil erreichen.
In konkreten Zahlen bedeutet das dann einen Absatz von 170 Turniere im zweiten Jahr.
Dieses Ziel wird für die kommenden fünf Jahre auch für die neu eingeführten Sportarten angestrebt, mit weiteren
	einsatzfähigen Systemen könnte im Laufe der nächsten fünf Jahre dann einen Absatz von 320 Turnieren jährlich
	erreicht werden.


\section{Kennzahlen}

Wie im Absatz~\ref{marktpotential} beschrieben lässt sich der Erfolg des Systems in Turnieren, die pro Jahr
durchgeführt werden, beschreiben.
Als Erfolgszhalen die wir und anfänglich erhoffen legen wir fest:
\begin{itemize}
	\item Mindestens 25 ausgerichteten Turnieren im ersten Jahr
	\item Mindestens einen jährlichen Zuwachs um 300\% in den folgenden zwei Jahren bezogen auf die Anzahl der
	Turniere im ersten Jahr
\end{itemize}

\section{Verkaufs- und Marketingkanäle}

Damit möglichst viele von unserem Produkt erfahren, wollen wir unsere Software direkt den Taekwondo-Vereinen und
den Veranstaltern von Taekwondo-Turnieren vorstellen.
Hierfür ist zusätzlich ein Flyer für potenzielle Veranstalter anzudenken.
Da es keine offiziellen Systeme und Informationsseiten in der Taekwondo-Community bezüglich solcher Bewertungssysteme
gibt, wird sehr viel durch Mundpropaganda bekannt gemacht.
So hat sich auch schon der Prototyp herumgesprochen und ist bereits für mehrere Online-Turniere gebucht worden.
\\\\
Bei neuen Sportarten würden wir ähnlich vorgehen und uns gezielt Leute suchen, welche schon eine Weile diese Sportart
	ausüben, so kann die Bewertungsapp optimal auf die spezifische Sportart angepasst werden.
Durch die dann entwickelte erste Version können bei Tests erste Kunden in anderen Sportarten gewonnen werden.
Zusätzlich zu der Vorstellung bei Vereinen bietet sich in bekannten Sportarten Werbung in Fachzeitschriften an.

\section{Kundengruppen}

Die Kundengruppe, für die das System entwickelt wird, sind alle Vereine, Obleute oder private Veranstalter von
Turnieren für die Sportarten Taekwondo, Eiskunstlauf, Geräteturnen und Gymnastik betreffen.
Sie sind dafür verantwortlich, dass das Datenmanagement der Sportler, die Bewertung und auch die schlussendliche
Auswertung der Punkte stabil und fehlerfrei vonstattengeht.
Das System Jeomsu kommt dabei in all diesen Bereichen zum Einsatz, sowhl von dem Turnierleiter oder von den
Schiedsrichtern.