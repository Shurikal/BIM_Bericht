\section{Marktpotential}\label{marktpotential}

Nach einem halben Jahr Entwicklungszeit ist ein erster funktionierender Prototyp vorhanden, dieser befindet sich aktuell in
der internen Testphase.
Sobald das interne Testing abgeschlossen ist, wird der Prototyp einer kleinen Kundegruppe zum Testen zur Verfügung gestellt.
Einzige Bedingung ist, dass die Kunden ein Feedback Formular ausfüllen welches zum Verbessern des Produkts genutzt
werden kann.
\\\\
%todo, Was hat das mit dem Marktpotential zu tun?
%Argumentation, warum das so ist und warum wir von diesen Zahlen ausgehen

%Nach einem halben Jahr Entwicklungszeit ist bereits ein erster stabiler Prototyp vorhanden, dieser befindet sich in
%der Testphase.
%Nach dem Testing ist das Ziel, im ersten Jahr soviele Einsätze und Feedback wie möglich zu erhalten, da nebenbei die Version
%1 in der Entwicklung fertiggestellt wird.

%todo, am besten schauen wir das zusammen an :)
%Feel free to change ;) passt aber auch zusammen
Innerhalb dieser Zeit beträgt die Marktkapazität daher alle Taekwondo-Turnieren in deutschsprachigen Ländern, das
sind ca. 80 Turniere pro Jahr.
Da es sich nur um ein kleines Team handelt ist es mit nur einem Prototyp-System schwierig, mehrere Turniere
	gleichzeitig abzuhalten, daher beschränkt sich das Marktpotential auf 52 Turniere pro Jahr, denn die meisten
	Turniere finden an einem Wochenende statt.
Es wäre wünschenswert, wenn wir in dieser Zeit einen Marktanteil von 30\% erreichen könnten, das entspricht 32
	Turnieren pro Jahr.
%Foto mit 1 Jahres marktanalyse einfügen oder referenzieren
\\\\
Mit der Vision das System auf mehrere Sportarten auszuweiten könnten schon im zweiten Jahr mehrere verschiedene
	Sportarten bewertet werden.
Recherchen haben ergeben, dass in den geplanten Sportarten Geräteturnen, Eiskunstlauf und Gymnastik ebenfalls pro
	Sportart ca. 100 Turniere jährlich stattfinden.
Die Marktkapazität erhöht sich somit auf das dreifache.
Mit mehreren einsatzbereiten Systemen können an einem Wochenende mehrere Turniere betreut werden.
So würde sich das Markpotential mit fünf einsatzbereiten System auf 260 Turniere erhöhen.
Durch die Neueinführung der zusätzlichen Sportarten im zweiten Geschäftsjahr rechnen wir ebenfalls mit einem
	Marktanteil von 30\% .
Mit dem zu diesem Zeitpunkt in der Taekwondo-Community bereits bekannten System könnten wir dort schon das Ziel von
	80\% Marktanteil erreichen.
In konkreten Zahlen bedeutet das dann einen Absatz von 170 Turniere im zweiten Jahr.
Dieses Ziel wird für die kommenden fünf Jahre auch für die neu eingeführten Sportarten angestrebt, mit weiteren
	einsatzfähigen Systemen könnte im Laufe der nächsten fünf Jahre dann einen Absatz von 320 Turnieren jährlich
	erreicht werden.
%Bild Marktanalyse jahr 2-5

\section{Kennzahlen}
% Welche messbaren Kennzahlen zeigen, dass die Lösung und das Ge-schäftsmodell funktionieren?

Wie im Absatz \ref{marktpotential} beschrieben lässt sich der Erfolg des Systems in Turnieren, die pro Jahr
durchgeführt werden, beschreiben.
Als Erfolgszhalen die wir und anfänglich erhoffen legen wir fest:
\begin{itemize}
	\item Mindestens 25 ausgerichteten Turnieren im ersten Jahr
	\item Mindestens einen jährlichen Zuwachs um 300\% in den folgenden zwei Jahren bezogen auf die Anzahl der
	Turniere im ersten Jahr
\end{itemize}

\section{Verkaufs- und Marketingkanäle}
% Über welche Kanäle werden die Kunden erreicht? Wie wird das Angebot vermarktet?
Damit möglichst viele von unserem Produkt erfahren, wollen wir unsere Software direkt den Taekwondo-Vereinen und
den Veranstaltern von Taekwondo-Turnieren vorstellen.
Hierfür ist zusätzlich ein Flyer für potenzielle Veranstalter anzudenken.
Da es keine offiziellen Systeme und Informationsseiten in der Taekwondo-Community bezüglich solcher Bewertungssysteme
gibt, wird sehr viel durch Mundpropaganda bekannt gemacht.
So hat sich auch schon der Prototyp herumgesprochen und ist bereits für mehrere Online-Turniere gebucht worden.
\\\\
Bei neuen Sportarten würden wir ähnlich vorgehen und uns gezielt Leute suchen, welche schon eine Weile diese Sportart
	ausüben, so kann die Bewertungsapp optimal auf die spezifische Sportart angepasst werden.
Durch die dann entwickelte erste Version können bei Tests erste Kunden in anderen Sportarten gewonnen werden.
Zusätzlich zu der Vorstellung bei Vereinen bietet sich in bekannten Sportarten Werbung in Fachzeitschriften an.

\section{Kundengruppen}
% Wer sind die wichtigsten Kundensegmente? Was sind deren Aufgaben (Job to be done)?
% Wo und wie wird das Produkt durch wen genutzt (An-wendungsfälle)? Was erzeugt dabei Frust (Pains) oder Lust (Gains)?
Die Kundengruppe, für die das System entwickelt wird, sind alle Vereine, Obleute oder private Veranstalter von
Turnieren für die Sportarten Taekwondo, Eiskunstlauf, Geräteturnen und Gymnastik betreffen.
Sie sind dafür verantwortlich, dass das Datenmanagement der Sportler, die Bewertung und auch die schlussendliche
Auswertung der Punkte stabil und fehlerfrei vonstattengeht.
Das System Jeomsu kommt dabei in all diesen Bereichen zum Einsatz, sowhl von dem Turnierleiter oder von den
Schiedsrichtern.


%\section{Markt und Differenzierung}
%Um zu sehen, ob unsere Geschäftsidee Anschluss im Markt finden würde, wurde zuerst das Marktpotential berechnet.

%Marktanalyse im Anhang

%Wie man sieht, sollte es auf dem Markt noch Platz haben für unser Produkt.
%Die Zahlen gelten nicht für das erste Jahr, sondern sind gross gerechnet, nach ca. 5 Jahren.
%In dieser Zeit sollte sich unser Produkt herumgesprochen und verbreitet haben.
%Wir rechnen damit, dass wir von Österreich her aus dem Taekwondo starten und unseren Einflussbereich Jahr für Jahr
%vergrössern.
%So sollte die App nach 5 Jahren europaweit bekannt und genutzt werden.
%Ausserdem wollen wir noch neue Sportarten hinzunehmen, beziehungsweise neue Bewertungsmöglichkeiten in unserer App anbieten.
%Das könnten z.B. Eiskunstlaufen, Geräteturnen und Gymnastik sein.
%\\
%Unsere Lösung zeichnet sich dadurch aus, dass sie sehr stabil und flexibel ist.
%Da sie plattformunabhängig ist, braucht man kein spezielles Gerät, sondern jeder Bewerter kann sein eigenes Handy
%dafür benutzen.
%So bleibt es unkompliziert und man kann sich die Mietkosten für die Hardware sparen.
%Falls es jedoch gewünscht ist, kann man diese auch direkt bei uns mitmieten.
%Dank verschiedenen Abos und dem App ist unser System auch sehr kostengünstig.
%Falls man sich trotz der einfachen Bedienung noch unsicher fühlt, bieten wir eine Supporthotline an und verschiedene Schulungen.
%Auch die Coronapandemie stellt für unsere Software kein Hindernis dar, denn es ist als Onlinesystem ausgeführt und so
%auch von jedem Zuhause aus bedienbar.
%\\\\
%Damit möglichst viele von unserem Produkt erfahren, wollen wir unsere Software direkt den Taekwondo-Vereinen und
%	den Veranstaltern von solchen Turnieren vorstellen.
%Da wir selbst jemanden haben aus einem solchen Verein, wollen wir die Software zuerst vor allem durch Mundpropaganda
%	bekannt machen.
%Eventuell auch die Flyer oder Werbung direkt an die Vereine schicken oder an Turnieren verteilen.
%So sollte sich unsere Lösung schnell herumsprechen und neue Kunden anlocken.
%Bei neuen Sportarten würden wir ähnlich vorgehen und uns gezielt Leute suchen, welche schon eine Weile diese Sportart
%	ausüben und sich darin auskennen.
%So kann die Bewertungsapp auf die spezifische Sportart angepasst werden.

%\subsection{Markt}
%Eine genaue Marktanalyse ist schwierig.
%Um genaue Informationen zu erhalten müsste man die einzelnen Kreis- und Landesverbände kontaktieren.
%Dies zeigt jedoch bereits, dass es bisher keine einheitliche und weithin bekannte Lösung gibt.
%Es existieren bisher Tools, um genaue Videoanalysen in Zeitlupe zu machen, wie sie beispielsweise in der Gymnastik
%	eingesetzt werden.
%Weiter gibt es Software, welche es erlaubt, Turnierbewertungen zu verwalten.
%Jedoch steht keine Onlinelösung zur Verfügung noch stellen sie Hardware zur Verfügung.
%\\
%Ausserdem ist davon auszugehen, dass bisherige Systeme sich auf eine Sportart beschränken.
%Unser System lässt sich im Taekwondo, Eiskunstlauf, Gymnastik wie auch Geräteturnen einsetzen und soll künftig auch
%	auf weitere Sportarten ausgeweitet werden.

%\subsection{Unfairer Vorteil}
%Wie sich im vergangenen Jahr gezeigt hat, können unvorhergesehene Ereignisse auftreten, welche Turnierveranstalter*innen zwingen, mit neuen Ideen aufzuwarten.
%Unser System kann sowohl für reguläre Turniere eingesetzt werden wie auch für Online-Turniere, wie sie in Pandemiezeiten stattfinden können.
%Dieses Feature ist ein entscheidender Vorteil gegenüber anderen Systemen.
%\\
%Des Weiteren wird unser System mit fundiertem Wissen im technischen wie auch im sportlichen Bereich entwickelt.
%Die sportlichen Kenntnisse beschränken sich im Moment auf Taekwondo.
%Es ist anzunehmen, dass dies als Grundlage ausreichend ist und für die Sportarten Eiskunstlauf, Gymnastik und Geräteturnen
%	nur geringfügige Modifikationen notwendig sind.